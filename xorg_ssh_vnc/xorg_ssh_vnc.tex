\documentclass[a4paper,11pt,notitlepage]{article}
\usepackage{polski}
\usepackage[utf8]{inputenc}
\usepackage{amsmath}
\usepackage{amsfonts}
\usepackage{amssymb}
\usepackage{geometry}
\usepackage{graphicx}

\geometry{verbose,a4paper,tmargin=1cm,bmargin=2cm,lmargin=2cm,rmargin=2cm}

\title{\textbf{Laboratorium sieci komputerowych\\Zajęcia wprowadzające}}
\author{Artur Skonecki\\Łukasz Załęski\\gr. 3}

\begin{document}
\maketitle
\begin{abstract}
\noindent
\large
Podczas zajęć wprowadzających zapoznaliśmy się ze sprzętem oraz technologiami stosowanymi w labolatorium w sali SK103. W trakcie zajęć zostały również przedstawione polecenia umożliwiające i ułatwiające pracę w labolatorium.
\end{abstract}



\section{Schemat}
\begin{verbatim}


   ____                                           ______
  |    | .187:x11                             .3 |      |
  | k7 |-----------------------------------------| volt |
  |____|            194.29.146.0/24              |______|


\end{verbatim}


\section{Uruchomienie klientów X na zdalnej maszynie.}


\subsection{Konfiguracja}
Po uruchomieniu maszyny \texttt{k7} i zalogowaniu się należało zamontować ramdysk jako system plików w katalogu domowym:
\begin{verbatim}
k6% home -r
Filesystem    Size    Used   Avail Capacity  Mounted on
/dev/md1       18M    8.0k     17M     0%    /home/stud/skonecka
\end{verbatim}
Teraz można było uruchomić X-serwer:
\begin{verbatim}
k7% startx
\end{verbatim}
\subsubsection{xauth}
Podczas uruchamiania środowiska graficznego w katalogu domowym powstał plik \texttt{.Xauthority} zawierający klucze autoryzujące dostęp do serwera.
\begin{verbatim}
k7% xauth list
k7.iem.pw.edu.pl/unix:0  MIT-MAGIC-COOKIE-1  7ab117fec480d35cf04795ba3a368a84
k7.iem.pw.edu.pl:0       MIT-MAGIC-COOKIE-1  7ab117fec480d35cf04795ba3a368a84
\end{verbatim}
Następnie wyeksportowaliśmy klucz do maszyny, na której miały być uruchamiane aplikacje klienckie X. Do tego celu wykorzystano skrypt \texttt{xauth-export}:
\begin{verbatim}
k7% xauth-export volt
przenosimy: k7.iem.pw.edu.pl:0
Password:
xauth:  creating new authority file /home/stud/skonecka/.Xauthority
-e na maszynie volt wykonaj polecenie:
export DISPLAY=k7.iem.pw.edu.pl:0
\end{verbatim}
Kolejnym krokiem było zalogowanie się na maszynę \texttt{volt} i odpowiednie ustawienie zmiennej środowiskowej \texttt{\$DISPLAY}:
\begin{verbatim}
k7% ssh volt
Password:
volt% export DISPLAY="k7.iem.pw.edu.pl:0"
\end{verbatim}
\subsubsection{xhost}
Alternatywą dla kluczy autoryzująych jest program xhost, który pozwala na kontrolę dostępu do serwera X poprzez nazwy hostów.
\begin{itemize}
\item Umożliwia wszystkim klientom z maszyny łączenie się z serwerem X
\begin{verbatim}
xhost +localhost
\end{verbatim}
\item Umożliwia wszystkim klientom z serwera volt łączenie się z serwerem X
\begin{verbatim}
xhost +volt
\end{verbatim}
\end{itemize}
Autoryzacja na poziomie hostów nadaje się jedynie do systemów z jednym użytkownikiem.


\subsection{Testy}
Sprawdzenie poprawności działania polegało na uruchomieniu dowolnego klienta korzystającej z X-serwera, np. \texttt{xterm}:
\begin{verbatim}
volt% xterm
\end{verbatim}
\section{Generowanie i instalacja kluczy ssh}
Poprzez wykorzystanie kryptografii asymetrycznej możliwe jest logowanie przez ssh bez podawania hasła. Pierwszym krokiem jest wygenerowanie pary kluczy prywatnego i publicznego:
\begin{verbatim}
volt% ssh-keygen -t rsa
\end{verbatim}
Po wykonaniu polecenia zostana wygenerowane 2 pliki
\begin{verbatim}
Zawierający klucz publiczny
$HOME/.ssh/id_rsa.pub
Oraz prywatny
$HOME/.ssh/id_rs
\end{verbatim}
N
Następnie należy dodać klucz publiczny do pliku \$HOME/.ssh/authorized\_keys na maszynie na którą będzie wykonywane logowanie:
\begin{verbatim}
volt% ssh-copy-id -i HOME/.ssh/id_rsa.pub k7
Password:
Now try logging into the machine, with "ssh 'k7'", and check in:

  ~/.ssh/authorized_keys

  to make sure we haven't added extra keys that you weren't expecting.
\end{verbatim}
Po wykonaniu tej instrukcji, powinno być możliwe wykonanie logowania z maszyny volt bez konieczności podawania hasła do konta na maszynie k7.

\section{Serwer VNC}
vncserver  jest  używany  do uruchomienia pulpitu VNC (Virtual Network 
Computing).
\subsection{Ustawienia startowe i u ruchomienie serwera VNC}
W pliku
\begin{verbatim}
$HOME/.vnc/xstartup
\end{verbatim}
Jest określone które aplikacje X zostaną uruchomione, przy starce vnc 
servera.
Np. Można zmienić managera okienek.
Server vnc uruchamiamy za pomocą polecenia:
\begin{verbatim}
vncserver
\end{verbatim}
Jeśli wszystko przebiegło pomyślnie, dostajemy komuniat o podobnej treści:
\begin{verbatim}
New '[host]:[id] ([user])' desktop is [host]:[id]
\end{verbatim}
Gdzie:
\begin{itemize}
\item [host] - nazwa hosta
\item [id] - id pulpitu
\item [user] - nazwa usera( domyślnie tego user, który włączył vnc server)
\end{itemize}
Każdy user może włączać wiele serwerów vnc, do połączenia się z różnymi 
vnc służy [id]\newline
\subsection{Nawiązanie połączenia z vncserver}
\subsubsection{Windows}
Aby połączyć się z serverem vnc, wystarczy pobrać, zainstalować i 
uruchomić klienta vnc np. TigerVNC Viewer. Obsługa takich programów 
zwykle ogranicza się do wpisania w pole server: [host]:[id], oraz 
podania nazwy użytkownika oraz hasła i wciśnięciu przycisku connect
\subsection{Linux}
Uruchomienie przeglądarki vnc w linuxie odbywa się za pomoca polecenia:
\begin{verbatim}
vncviewer [host]:[id]
\end{verbatim}
Po uruchomieniu powyższego polecenia, otworzy się okno przeglądarki z 
dialogiem do wpisania usera i hasła
\subsection{Zmiana hasła dla vnc}
\begin{verbatim}
vncpasswd
\end{verbatim}
Plecenie wypytuje nas o nowe hasło

\pagebreak

\section{Wnioski}
X serwer to aplikacja, do której zadań należy udostępnianie klawiatury, myszy i wyświetlacza pracującego w trybie graficznym. Aplikacje komunikujące się z X serwerem to X klienty. Interakcja X klientów z serwerem X przebiega za pomocą sieciowo transparentnego protokołu komunikacyjnego, dzięki czemu X klienty mogą pracować na zupełnie innych maszynach. Rozszerzenie MIT-SHM dla systemu okien X dodaje możliwość komunikacji między serwerem a klientami przez pamięć współdzieloną, co znacznie poprawia wydajność.


Poznane komendy.
\begin{itemize}
    \item killall
    \item rdesktop, xtsc
    \item lpr, a2ps
    \item ssh, ssh-keygen, ssh-copy-id
    \item home
    \item xauth, xauth-export
    \item startx
    \item xhost
    \item scp
    \item vncserver, vncviewer
    \item script
\end{itemize}
\end{document}
