\documentclass[a4paper,11pt,notitlepage]{article}
\usepackage[utf8]{inputenc}
\usepackage[T1]{fontenc}
\usepackage[polish]{babel}
\selectlanguage{polish}
\usepackage{graphicx}

\usepackage{geometry}

%\hyphenation{FreeBSD,ppp,PPP}


\geometry{verbose,a4paper,tmargin=1cm,bmargin=2cm,lmargin=2cm,rmargin=2cm}

\title{\textbf{Laboratorium sieci komputerowych\\Konfiguracja mostka, protokół ARP, konfiguracja PPP}}
\author{Artur Skonecki\\Łukasz Załęski\\gr. 3}


\usepackage{indentfirst}

\begin{document}
\maketitle
%\tableofcontents

\section{Schematy}

\subsection{Schemat nullmodem}
\begin{verbatim}
   ____                                           ____
  |    | tun0 .1      null modem         tun0 .2 |    |
  | k7 |-----------------------------------------| k8 |
  |____|              10.17.17.0/32              |____|
\end{verbatim}
\subsection{Schemat mostek}
\begin{verbatim}
   ____                                           ____
  |    | wlan0 .1     adhoc K7_K8       wlan0 .2 |    |
  | k7 |-----------------------------------------| k8 |
  |____|              10.17.17.0/24              |____|
  bridge0 -> sk0
\end{verbatim}

\section{ARP}
\subsection{Wyświetlenie tablicy ARP}
Wyświetlamy tablicę arp na k8 za pomocą: arp -n -a
\begin{verbatim}
k8% arp -n -a
? (194.29.146.1) at 00:14:1c:ad:7b:43 on sk0 expires in 1127 seconds [ethernet]
? (194.29.146.3) at 00:e0:81:65:37:d4 on sk0 expires in 1121 seconds [ethernet]
\end{verbatim}
\subsection{Wykasowanie tablicy ARP}
Aby sprawdzić dalsze zachowanie arp wyczyściliśmy tablicę arp za pomocą 
polecenia: arp -d -a
\begin{verbatim}
k8% sudo arp -d -a
194.29.146.1 (194.29.146.1) deleted
194.29.146.3 (194.29.146.3) deleted
\end{verbatim}
Teraz ponownie sprawdzamy zawartość tablicy arp:
\begin{verbatim}
k8% arp -n -a
k8% arp -n -a
? (194.29.146.3) at 00:e0:81:65:37:d4 on sk0 expires in 1195 seconds 
[ethernet]
? (194.29.146.188) at 00:11:d8:4a:c5:cc on sk0 permanent [ethernet]
\end{verbatim}
\subsection{Nasłuchiwanie zapytań ARP}
Teraz podglądamy jak wygląda zapytanie arp kierowanie do k8 za pomocą: 
tcpdump
\begin{verbatim}
k8% sudo tcpdump -c 4 -n arp
listening on sk0, link-type EN10MB (Ethernet), capture size 65535 bytes
22:24:27.957386 ARP, Request who-has 192.168.2.70 tell 192.168.2.70, 
length 46
22:24:28.670615 ARP, Request who-has 192.168.123.118 tell 192.168.146.3, 
length 46
\end{verbatim}
\section{Ustanowienie połączenia  przez kabel RS-232 (null modem) z wykorzystaniem protokołu PPP}


Po połączeniu 2 komputerów za pomocą kabla RS-232, wykonaliśmy konfigurację programu ppp poprzez edycję pliku "/etc/ppp/ppp.conf".
Na maszynie k7 w sekcji \textbf{nullmodem} pliku konfiguarcyjnego:

\begin{verbatim}
set ifaddr 10.17.17.1 10.17.17.2  255.255.255.255
\end{verbatim}

Na maszynie k8 w sekcji \textbf{nullmodem} pliku konfiguarcyjnego:

\begin{verbatim}
set ifaddr 10.17.17.2 10.17.17.1  255.255.255.255
\end{verbatim}

Następnie uruchomiliśmy ppp.

\begin{verbatim}
sudo ppp nullmodem
\end{verbatim}

interfejs tun0 - maszyna k7
\begin{verbatim}
tun0: flags=8051<UP,POINTOPOINT,RUNNING,MULTICAST> metric 0 mtu 1500
	options=80000<LINKSTATE>
	inet 10.17.17.1 --> 10.17.17.2 netmask 255.255.255.255
	Opened by PID 937
\end{verbatim}

interfejs tun0 - maszyna k8

\begin{verbatim}
tun0: flags=8051<UP,POINTOPOINT,RUNNING,MULTICAST> metric 0 mtu 1500
	options=80000<LINKSTATE>
	inet 10.17.17.2 --> 10.17.17.1 netmask 255.255.255.255
	Opened by PID 7681
\end{verbatim}

Przetestowaliśmy połączenie obustronnie przy użyciu tcpdump i ping.

\begin{verbatim}
k7% ping -c 1 -S 10.17.17.1 10.17.17.2
PING 10.17.17.2 (10.17.17.2) from 10.17.17.1: 56 data bytes
64 bytes from 10.17.17.2: icmp_seq=0 ttl=64 time=8.555 ms
\end{verbatim}
\begin{verbatim}
k8% sudo tcpdump -i tun0
listening on tun0, link-type NULL (BSD loopback), capture size 65535 bytes
13:05:22.313695 IP un-registered-LAN10 > un-registered-LAN10: ICMP echo request, id 13469, seq 0, length 64
13:05:22.313729 IP un-registered-LAN10 > un-registered-LAN10: ICMP echo reply, id 13359, seq 0, length 64
\end{verbatim}

\section{Konfiguracja mostka}
Ćwiczenie polegało na konfiguracji mostka (bridge) i uruchomieniu protokołu STP (Spanning Tree Protocol).



Pomiędzy maszynami k7 i k8 utworzyliśmy połączenie typu adhoc.
\begin{verbatim}
k7% sudo ifconfig wlan create wlandev rum0 wlanmode adhoc ssid 'K7_K8'
k7% sudo ifconfig wlan0 inet 10.17.17.1/24 up
\end{verbatim}
\begin{verbatim}
k8% sudo ifconfig wlan create wlandev ural0 wlanmode adhoc ssid 'K7_K8'
k8% sudo ifconfig wlan0 inet 10.17.17.2/24 up
\end{verbatim}

Utworzenie mostka i właczenie STP:
\begin{verbatim}
k7% sudo ifconfig bridge create
k7% sudo ifconfig bridge0 addm sk0 addm wlan0
k7% sudo ifconfig bridge0 stp sk0 stp wlan0
k7% sudo ifconfig bridge0 up
\end{verbatim}

Następnie sprawdziliśmy ustawienia mostka:

\begin{verbatim}
k7% ifconfig bridge0
bridge0: flags=8802<BROADCAST,SIMPLEX,MULTICAST> metric 0 mtu 1500
	ether 02:7d:1d:ad:cc:00
	id 00:0f:cb:ff:13:a7 priority 32768 hellotime 2 fwddelay 15
	maxage 20 holdcnt 6 proto rstp maxaddr 2000 timeout 1200
	root id 00:14:1c:ad:7b:00 priority 16388 ifcost 220000 port 1
	member: wlan0 flags=147<LEARNING,DISCOVER,STP,AUTOEDGE,AUTOPTP>
	        ifmaxaddr 0 port 5 priority 128 path cost 370370 proto rstp
	        role designated state discarding
	member: vr0 flags=1c7<LEARNING,DISCOVER,STP,AUTOEDGE,PTP,AUTOPTP>
	        ifmaxaddr 0 port 1 priority 128 path cost 200000 proto rstp
	        role root state discarding
\end{verbatim}

Sprawdziliśmy działanie mostka za pomocą tcpdump.

\begin{verbatim}
k7% sudo tcpdump -i wlan0
22:38:12.776351 STP 802.1w, Rapid STP, Flags [Learn, Forward], bridge-id 8000.K7-W.8007, length 36
22:38:13.423461 ARP, Request who-has 192.168.2.70 tell 192.168.2.70, length 46

k8% sudo tcpdump -i wlan0
22:42:23.449938 ARP, Request who-has 192.168.2.70 tell 192.168.2.70, length 46
22:42:23.502402 IP 10.146.5.8.netbios-ns > 10.146.255.255.netbios-ns: NBT UDP PACKET(137): QUERY; REQUEST; BROADCAST
\end{verbatim}


\section{Podsumowanie}
\subsection{Podsumowanie ARP}
Ćwiczenie polegało na diagnostyce protokołu ARP. Protokół ARP jest 
używany do tłumaczenia adresów internetowych na
adresy sprzętowe używane przez sieci lokalne. ARP może pracować z tymi 
rodzajami sieci, które posiadają mechanizm rozgłaszania.
Zasada działania ARP polega na wymianie komunikatów żądanie - odpowiedź. 
Węzeł szukający adresu sprzętowego związanego
z określonym adresem IP rozgłasza pakiet "Żądanie ARP". każdy węzeł 
dołączony do sieci odbiera taki komunikat i jeśli rozpozna,
że szukany jest jego adres IP - zapisuje w pamięci adres sprzętowy i IP 
nadawcy i wysyła pakiet "Odpowiedź ARP".



\subsection{Podsumewanie PPP RS-232}
PPP (Point to Point Protocol) to protokół warstwy łącza. Pozwala na ustanowienie bezpośredniego połączenia między węzłami sieci komputerowej. PPP jest często wykorzystywany przez dostawców usługi internetowej do zapewniania uwierzytelniania.

\subsection{Podsumowanie mostek}
Mostek pracuje na poziomie warstwy łącza danych. Przekazuje pakiety wykorzystując adresy MAC. W tym celu wykorzystuje tablicę mostowania, w której o segmentach sieci skojarzonych z adresami MAC. W celu znalezienia adresów MAC, które nie znajdują się w tablicy forwardingu, mostek zalewa sieć przekazując pakiet na wszystki porty z wyjątkiem źródłowego.

STP (Spanning Tree Protocol) to protokół warstwy łącza danych. Pozwala on na wykrywanie i blokowanie pętli w sieci oraz ustalanie zapasowych łącz. Obecnie dominuje wstecznie zgodna wersja protokołu STP - RSTP (Rapid Spanning Tree Protocol). Charakteryzuje się ona szybszym wykrywaniem pętli i uruchamianiem zapasowych łacz.


\end{document}
