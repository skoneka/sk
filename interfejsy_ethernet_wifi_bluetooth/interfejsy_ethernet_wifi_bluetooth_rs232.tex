\documentclass[a4paper,11pt,notitlepage]{article}
\usepackage{polski}
\usepackage[utf8]{inputenc}
\usepackage{amsmath}
\usepackage{amsfonts}
\usepackage{amssymb}
\usepackage{geometry}
\usepackage{graphicx}

\geometry{verbose,a4paper,tmargin=1cm,bmargin=2cm,lmargin=2cm,rmargin=2cm}

\title{\textbf{Laboratorium sieci komputerowych\\Konfiguracja interfejsów i łącz sieciowych}}
\author{Artur Skonecki\\Łukasz Załęski\\gr. 3}

\begin{document}
\maketitle
\begin{abstract}
\noindent
\large
Celem zajęć było zapoznanie się z konfiguracją interfejsów i łącz sieciowych.
\end{abstract}


\section{Ethernet}

\begin{verbatim}
   ____                                           ______
  |    | .16.16                             .7.3 |      |
  | k7 |-----------------------------------------| volt |
  |____| fxp0         10.146.0.0/16        lagg0 |______|
\end{verbatim}

\subsection{Konfiguracja}

W celu uniknięcia konfliktów, za pomocą komendy ping sprawdziliśmy czy adres IP nie jest zajęty.

\begin{verbatim}
k7% ping -c1 10.146.16.16
1 packets transmitted, 0 packets received, 100.0% packet loss
\end{verbatim}

Następnie załadowaliśmy sterownik do drugiej karty ethernet
\begin{verbatim}
kldload ip_fxp
\end{verbatim}
Oraz ustawiliśmy adres IP i maskę na interfejsie na fxp0.
\begin{verbatim}
sudo ifconfig fxp0 inet 10.146.16.16/16 up
\end{verbatim}


\begin{verbatim}
k7% ifconfig fxp0
fxp0: flags=8843<UP,BROADCAST,RUNNING,SIMPLEX,MULTICAST> metric 0 mtu 1500
        options=4219b<RXCSUM,TXCSUM,VLAN_MTU,VLAN_HWTAGGING,VLAN_HWCSUM,TSO4,WOL_MAGIC,VLAN_HWTSO>
        ether 00:0e:0c:00:fc:05
        inet 10.146.16.16 netmask 255.255.0.0 broadcast 10.146.255.255
        media: Ethernet autoselect (100baseTX <full-duplex>)
        status: active
\end{verbatim}

W celu przetestowania połączenia wysłaliśmy pakiet ICMP z volta.
\begin{verbatim}
$ ping -c1 10.146.16.16
PING 10.146.16.16 (10.146.16.16): 56 data bytes
64 bytes from 10.146.16.16: icmp_seq=0 ttl=64 time=0.507 ms
\end{verbatim}

Ponadto, nasłuchiwaliśmy pakiety ICMP za pomocą komendy tcpdump.
\begin{verbatim}
k7% sudo tcpdump -i fxp0 icmp
20:41:09.371839 IP 10.146.7.3 > 10.146.16.16: ICMP echo request, id 38667, seq 0, length 64
20:41:09.371872 IP 10.146.16.16 > 10.146.7.3: ICMP echo reply, id 38667, seq 0, length 64
\end{verbatim}
\subsection{Uzyskanie adresu za pomocą DHCP}
Następnie ponownie skonfigurowaliśmy interfejs fxp0 za pomocą klienta DHCP.
\begin{verbatim}
k7% sudo dhclient fxp0
DHCPDISCOVER on fxp0 to 255.255.255.255 port 67 interval 4
DHCPOFFER from 10.146.7.3
unknown dhcp option value 0xaf
DHCPREQUEST on fxp0 to 255.255.255.255 port 67
DHCPACK from 10.146.7.3
unknown dhcp option value 0xaf
bound to 10.146.226.102 -- renewal in 1800 seconds.
\end{verbatim}
W celu przetestowania połączenia wysłaliśmy pakiet ICMP z volta do maszyny k7.
\begin{verbatim}
$ ping -c1 10.146.226.102
PING 10.146.226.102 (10.146.226.102): 56 data bytes
64 bytes from 10.146.226.102: icmp_seq=0 ttl=64 time=0.287 ms
\end{verbatim}
Na maszynie k7 tcpdump przechwycił pakiet.
\begin{verbatim}
k7% sudo tcpdump  -i fxp0 icmp
20:53:05.193131 IP 10.146.7.3 > 10.146.226.102: ICMP echo request, id 12568, seq 0, length 64
20:53:05.193159 IP 10.146.226.102 > 10.146.7.3: ICMP echo reply, id 12568, seq 0, length 64
\end{verbatim}
\section{WiFi}
\subsection{Połączenie z ZETiIS}
Załadowanie sterowników i utworzenie interfejsu wlan0.
\begin{verbatim}
sudo kldload uhci ehci
sudo kldload if_rum
sudo ifconfig wlan create wlandev rum0
\end{verbatim}

\begin{verbatim}
k7% sudo ifconfig wlan0 up scan
SSID/MESH ID    BSSID              CHAN RATE   S:N     INT CAPS
ZETiIS          00:1e:52:79:ca:59    6   54M -75:-95  100 EPS  RSN HTCAP WME
\end{verbatim}

Wybranie AP ZETiIS.

\begin{verbatim}
sudo ifconfig wlan0 ssid ZETiIS
\end{verbatim}

Po edycji pliku /etc/wpa\_supplicant.conf (ustawienie loginu i hasła) uruchomiliśmy wpa\_supplicant.

\begin{verbatim}
sudo wpa_supplicant -iwlan0 -c/etc/wpa_supplicant.conf
\end{verbatim}

Połączenie zostało nawiązane. Stacja k7 odbierała pakiety ARP rozgłaszane jako broadcast.

\begin{verbatim}
k7% sudo tcpdump -i wlan0
tcpdump: WARNING: wlan0: no IPv4 address assigned
21:59:49.879058 EAP packet (0) v2, len 5
21:59:50.020294 EAP packet (0) v2, len 4
21:59:50.022169 EAPOL key (3) v2, len 117
21:59:50.027094 EAPOL key (3) v1, len 95
21:59:50.208516 IP s04.iem.pw.edu.pl.netbios-dgm > ZET-NET.iem.pw.edu.pl.netbios-dgm: NBT UDP PACKET(138)
21:59:50.310614 ARP, Request who-has argo tell vol2, length 46
21:59:50.675298 ARP, Request who-has 192.168.2.70 tell 192.168.2.70, length 46
\end{verbatim}
\subsection{Połączenie peer-to-peer}

Utworzenie interfejsu wlan0 w trybie adhoc.

\begin{verbatim}
sudo ifconfig wlan0 destroy
sudo ifconfig wlan0 create wlandev rum0 wlanmode adhoc
\end{verbatim}

Ustawienie SSID, adresu IP i maski.

\begin{verbatim}
sudo ifconfig wlan0 ssid K7
sudo ifconfig wlan0 inet 10.7/16
sudo ifconfig wlan0 up
\end{verbatim}

Wybraliśmy maszynę k5 w celu przetestowania połączenia z K7.
\begin{verbatim}
k5% sudo ifconfig wlan0 up scan
SSID/MESH ID    BSSID              CHAN RATE   S:N     INT CAPS
K7              2a:22:6c:6e:06:fa   10   54M -78:-96  100 IS
\end{verbatim}


Ustawienie połączenia na komputerze k5.
\begin{verbatim}
sudo ifconfig wlan create wlandev ath0 wlanmode adhoc
sudo ifconfig wlan0 ssid K7
sudo ifconfig wlan0 inet 10.5/16
\end{verbatim}

\begin{verbatim}
k5% ifconfig wlan0
wlan0: flags=8843<UP,BROADCAST,RUNNING,SIMPLEX,MULTICAST> metric 0 mtu 1500
        ether 00:0f:cb:ff:13:a7
        inet 10.0.0.5 netmask 255.255.0.0 broadcast 10.0.255.255
        media: IEEE 802.11 Wireless Ethernet autoselect mode 11g <adhoc>
        status: running
        ssid K7 channel 10 (2457 MHz 11g) bssid 2a:22:6c:6e:06:fa
        country US ecm authmode OPEN privacy OFF txpower 22.5 scanvalid 60
        protmode CTS wme burst
\end{verbatim}

Test połączenia.
\begin{verbatim}
k5% ping -c1 10.0.0.7
PING 10.0.0.7 (10.0.0.7): 56 data bytes
64 bytes from 10.0.0.7: icmp_seq=0 ttl=64 time=1.728 ms
\end{verbatim}

\section{Bluetooth}
\subsection{Wczytanie sterowników}
Załadowanie sterowników usb, bluetooth:
\begin{verbatim}
kldload uhci ehci
kldload ng_ubt
\end{verbatim}
\subsection{Uruchomienie systemu bluetooth}
Urochomienie systemu bluetooth dla urządzenia ubt0(modem bluetooth)
 \begin{verbatim}
 #/etc/rc.d/bluetooth [start|stop|restart|rcvar] ubt0
/etc/rc.d/bluetooth start ubt0
\end{verbatim}
\subsection{Uzyskanie dlisty dostepnych urządzeń}

Pobranie listy dostepnych urzadzeń bluetooth do podłączenia:

\begin{verbatim}
k6# hccontrol -n ubt0hci inquiry
\end{verbatim}

Polecenie zwróciło listę widocznych urządzeń, wraz z ich adresami lub aliasami.Aliasy znajduję się w pliku /etc/bluetooth/hosts
Uzyskanie nazwy urzędzenia po jego adresie:

\begin{verbatim}
k6# hccontrol -n ubt0hci remote_name_request 00:08:1b:00:d3:3a
BD_ADDR: K8
Name: k8 (ubt0)
\end{verbatim}

\subsection{Podłączenie k6 do k8}
Włączenie trybu punktu dostępowego(na k8 też ładujemy sterowniki i uruchamiamy system bluetooth):
\begin{verbatim}
k8# bt-pan -s
\end{verbatim}
Po wykonaniu polecenia, konfigurujemy interfejs tap0
\begin{verbatim}
k8# ifconfig tap0 10.8
k8# ifconfig tap0
tap0: flags=8843<UP,BROADCAST,RUNNING,SIMPLEX,MULTICAST> metric 0 mtu 1500
        options=80000<LINKSTATE>
        ether 00:08:1b:00:d3:3a
        inet 10.0.0.8 netmask 255.0.0.0 broadcast 10.255.255.255
\end{verbatim}

Następnie podłączamy k6 do punktu dostępowego:

\begin{verbatim}
k6# bt-pan -c k8
\end{verbatim}
Po czym konfigurujemy tap0 na k6
\begin{verbatim}
k8# ifconfig tap0 10.8
\end{verbatim}
Po wykonaniu powyższych operacji sprawdzamy połączenie z k8
\begin{verbatim}
k6# ping -a k8
44 bytes from K8 seq_no=0 time=38.798 ms result=0
\end{verbatim}

\section{Łącze szeregowego RS-232}
\subsection{Konfiguracja}
Ponieważ sterowniki były domyślnie wbudowane w kernel FreeBSD,
nie było koniczne wykonanie polecenia:
\begin{verbatim}
kldload uart
\end{verbatim}
Znalezienie parametrów łącza obsługiwanych przez obydwa porty.
\begin{verbatim}
tail /etc/gettytab
\end{verbatim}

\subsection{Połączenie}
Wykonanie na obydwu połączonych maszynach polecenia cu (call unix).
\begin{verbatim}
k7% sudo cu -l /dev/cuau0 -s 19200
\end{verbatim}
\begin{verbatim}
k8% sudo cu -l /dev/cuau0 -s 19200
\end{verbatim}
\subsection{Test}
Test łącza polegał na wpisaniu ciągu znaków, który był przesyłany na drugi komputer.
Zakończenie połączenia za pomocą sekwencji (po dodaniu opcji ``EscapeChar none'' do \textasciitilde/.ssh/config):
\begin{verbatim}
^M.~
\end{verbatim}
\section{Wnioski}
Dzięki temu ćwiczeniu zapoznaliśmy się z konfiguracją interfejsów i łącz sieciowych, wykorzystująć różne media transmisyjne.
Przeprowadziliśmy konfigurację połączeń za pomocą:
\begin{itemize}
    \item Ethernetu
    \item WiFi
    \item Bluetooth
    \item Łącza szeregowego RS-232
    \item Firewire
\end{itemize}
Podczas kładzenia warstwy sieciowej na interfejsy, ważne było wybranie adresu IP z puli adresów prywatnych oraz upewnienie się, że dany adres nie jest już wykorzystywany.
W systemie FreeBSD nazwy interfejsów sieciowych są tworzone w zależności od sterownika, który jest używany, co pozwala na łatwiejszą identyfikację fizycznych interfejsów.
Ćwiczenie pozwala dostrzec zalety łączy bezprzewodowych, w których wyeliminowanie kabla, jako medium transmisyjnego, pozwala na wygodniejsze
i sprawniejsze łączenie urządzeń. W przypadku braku połączenia nie musimy się zastanawiać, czy przypadkiem kabel nie jest uszkodzony.

\end{document}
