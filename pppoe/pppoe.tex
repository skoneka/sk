\documentclass[a4paper,11pt]{article}
\usepackage{polski}
\usepackage[utf8]{inputenc}
\usepackage{amsmath}
\usepackage{amsfonts}
\usepackage{amssymb}
\usepackage{geometry}

\geometry{verbose,a4paper,tmargin=1cm,bmargin=2cm,lmargin=2cm,rmargin=2cm}

\title{\textbf{Laboratorium sieci komputerowych\\PPPoE}}
\author{Artur Skonecki\\Łukasz Załęski}

\begin{document}
\maketitle

\section*{PPPoE - Point to Point Protocol over Ethernet}

\subsection*{Schematy}
Połączenie PPPoE zostało przetestowane za pomocą wirtualnych maszyn oraz maszyn znajdujących się w labolatorium SK103.


\begin{verbatim}
   ____                                           ____
  |    |-|tun0 10.0.0.1         10.0.1.234 tun0|-|    |
  |    | |                                     | |    |
  |    | |em3 .4                         .3 em3| |    |
  | mfs|-----------------------------------------| vf9|
  |____|             192.168.56.0/24             |____|
(serwer) (VirtualBox host-only adapter vboxnet0) (klient)

\end{verbatim}

\begin{verbatim}
                             194.29.146.0/24
     _________________________________________________________________
     |sk0 .247                            sk0 .248 |         bge1 .3 |
   __|_                                           _|__               |
  |    |-|tun0 10.0.0.1          10.0.4.40 tun0|-|    |           ___|__
  |    | |                                     | |    |          |      |
  |    | |em0 .226.222             .226.175 em0| |    |          | volt |
  | k1 |-----------------------------------------| k2 |          |______|
  |____|             10.146.0.0/16               |____|
(serwer)                                         (klient)


\end{verbatim}

\subsection*{Konfiguracja serwera PPPoE}

\begin{verbatim}
cat > /etc/ppp/ppp.conf << EOF
# http://www.freebsdonline.com/content/view/29/45/
# ppp.conf: PPPoE server configuration
default:
 set log +debug
 set device PPPoE:em3
 enable pap                             #turn on chap and pap accounting
 enable chap
 allow mode direct                      #turn on ppp bridging
 enable proxy                           #turn on ppp proxyarping (redundant of a
 disable ipv6cp                         #we don't use ipv6, don't want the errors
 set mru 1492                           #set mru below 1500 (PPPoE MTU issue)
 set mtu 1492                           #set mtu below 1500 (PPPoE MTU issue)
 set ifaddr 10.0.0.1 10.0.1.1-10.0.5.254
 set speed sync
 set timeout 0
 enable lqr
 accept dns
EOF

cat > /etc/ppp/ppp.secret << EOF
username	password
EOF

cat >> /etc/rc.conf << EOF
pppoed_enable="YES"
pppoed_flags='-Fd -P /var/run/pppoed.pid -a "server" -l "default"'
pppoed_interface="em3"
EOF
\end{verbatim}



\subsection*{Konfiguracja klienta PPPoE}
\begin{verbatim}
cat > /etc/ppp/ppp.conf << EOF
# http://renaud.waldura.com/doc/freebsd/pppoe/
# ppp.conf: PPPoE client configuration
default:
 set device PPPoE:em3
 set speed sync
 set mru 1492 
 set mtu 1492
 set ctsrts off
 # monitor line quality 
 enable lqr
 set log +debug
 # insert default route upon connection
 add default HISADDR
 enable dns

papchap:
 set authname username
 set authkey password
EOF
\end{verbatim}

\subsection*{Logowanie na serwerze}
\begin{verbatim}
cat >> /etc/syslog.conf << EOF
!ppp
*.*		/var/log/ppp.log
!pppoed
*.*		/var/log/pppoed.log
EOF
\end{verbatim}

\subsection*{Logowanie na kliencie}
\begin{verbatim}
cat >> /etc/syslog.conf << EOF
!ppp
*.*		/var/log/ppp.log
EOF
\end{verbatim}

\subsection*{Połączenie}

Przed połączeniem ppp konieczne było ustanowienie adresu IP na interfejsach ethernetowskich.
\begin{verbatim}
server% dhclient em3
client% dhclient em3
\end{verbatim}

\begin{verbatim}
server% /etc/rc.d/pppoed start
Starting pppoed
Sending NGM_LISTHOOKS to em3:
Got reply from id [4]: Type ether with 1 hooks
  Got [4]:orphans -> [6]:ethernet
Sending PPPOE_LISTEN to .:pppoe-1771, provider *
pppoed[1771]: Listening as provider *
pppoed[1771]: Got 60 bytes of data: ffffffffffff0800276e3787886311090000000c010300040061f1c10101000000000000000000000000000000000000000000000000000000000000
pppoed[1773]: Creating a new socket node
pppoed[1771]: Listening as provider *
pppoed[1773]: Sending CONNECT from .:exec-1773 -> em3:orphans.exec-1773
pppoed[1773]: Sending NGM_SOCK_CMD_NOLINGER to socket
pppoed[1773]: Offering to .:exec-1773 as access concentrator "server"
pppoed[1773]: adding to .:exec-1773 as offered service "server"
pppoed[1773]: Sending original request to .:exec-1773 (60 bytes)
pppoed[1773]: Waiting for a SUCCESS reply .:exec-1773
pppoed[1773]: Received NGM_PPPOE_SESSIONID (hook "")
pppoed[1773]: Received NGM_PPPOE_SUCCESS (hook "exec-1773")
pppoed[1773]: Executing: exec /usr/sbin/ppp -direct "default"
\end{verbatim}
\begin{verbatim}
client% ppp papchap
Working in interactive mode
Using interface: tun0
ppp ON Vfreebsd9> dial
ppp ON Vfreebsd9>
Ppp ON Vfreebsd9>
PPp ON Vfreebsd9> Warning: 0.0.0.0/0: Change route failed: errno: No such process
PPP ON Vfreebsd9> Warning: 0.0.0.0/0: Change route failed: errno: No such process
ppp ON Vfreebsd9>
\end{verbatim}


Zmieniająca się wielkość liter w ciagu znaków ``ppp'', powiadamia o powodzeniu poszególnych faz ustanawiania połączenia:
\begin{itemize}
\item Ppp - powiodła się faza inicjalizacji LCP (Link-Control Protocol)
\item PPp - udane uwierzytelnianie
\item PPP - został przyporządkowany adres IP
\end{itemize}


\begin{verbatim}
server% ifconfig tun0
tun0: flags=8051<UP,POINTOPOINT,RUNNING,MULTICAST> metric 0 mtu 1492
        options=80000<LINKSTATE>
        inet 10.0.0.1 --> 10.0.1.234 netmask 0xffffffff
        nd6 options=21<PERFORMNUD,AUTO_LINKLOCAL>
        Opened by PID 1773
\end{verbatim}

\begin{verbatim}
client% ifconfig tun0
tun0: flags=8050<POINTOPOINT,RUNNING,MULTICAST> metric 0 mtu 1500
        options=80000<LINKSTATE>
        inet 10.0.1.234 --> 10.0.0.1 netmask 0xffffffff
        nd6 options=29<PERFORMNUD,IFDISABLED,AUTO_LINKLOCAL>
        Opened by PID 2449
\end{verbatim}

Log tcpdump zawierający odpowiedź na pakiet ICMP zarejestrowaną po wykonaniu komendy "ping 10.0.0.1" na kliencie.
\begin{verbatim}
client% tcpdump -i em3
23:17:10.849905 PPPoE PADI [Host-Uniq 0x0061F1C1] [Service-Name]
23:17:10.868032 PPPoE PADO [AC-Name ""server""] [Service-Name] [Service-Name "*"] [Host-Uniq 0x0061F1C1] [AC-Cookie 0x40B2F1C3]
23:17:10.868422 PPPoE PADR [Host-Uniq 0x0061F1C1] [AC-Cookie 0x40B2F1C3] [AC-Name ""server""] [Service-Name]
23:17:10.869526 PPPoE PADS [ses 0x1] [AC-Name ""server""] [Service-Name] [Host-Uniq 0x0061F1C1] [AC-Cookie 0x40B2F1C3]
23:17:12.288443 PPPoE  [ses 0x1] LCP, Conf-Request (0x01), id 1, length 29
        encoded length 27 (=Option(s) length 23)
...<snip>...
23:17:32.910336 PPPoE  [ses 0x1] compressed PPP data
23:17:42.628901 PPPoE  [ses 0x1] LQM
        0x0000:  6890 a1d4 0000 0001 0000 0003 0000 006a
        0x0010:  0000 0001 0000 0003 0000 0000 0000 0000
        0x0020:  0000 006a 0000 0002 0000 000d 0000 014f
...<snip>...
23:20:13.171230 PPPoE  [ses 0x1] compressed PPP data
23:20:13.182320 PPPoE  [ses 0x1] IP (tos 0x0, ttl 64, id 179, offset 0, flags [none], proto ICMP (1), length 84)
    10.0.0.1 > 10.0.1.234: ICMP echo reply, id 60937, seq 0, length 64
\end{verbatim}

\begin{verbatim}
client% ping 10.0.0.1
64 bytes from 10.0.0.1: icmp_seq=0 ttl=64 time=35.000 ms
64 bytes from 10.0.0.1: icmp_seq=1 ttl=64 time=19.487 ms
\end{verbatim}

\begin{verbatim}
server% ping 10.0.2.234
64 bytes from 10.0.2.234: icmp_seq=0 ttl=64 time=10.966 ms
64 bytes from 10.0.2.234: icmp_seq=1 ttl=64 time=20.986 ms
\end{verbatim}


\subsection*{Automatyczna konfiguracja podczas startu systemu}

\subsubsection*{Serwer}
\begin{verbatim}
cat >> /etc/rc.conf << EOF
ifconfig_em3="DHCP"
pppoed_enable="YES"
pppoed_flags='-P /var/run/pppoed.pid -a "server" -l "default"'
pppoed_interface="em3"
EOF
\end{verbatim}

\subsubsection*{Klient}

\begin{verbatim}
cat >> /etc/rc.conf << EOF
ifconfig_em3="DHCP"
ppp_enable="YES"
ppp_mode="background"
ppp_profile="papchap"
EOF
\end{verbatim}
\subsection*{Podsumowanie}
PPPoE jest najczęściej wykorzystywany w technologiach DSL. PPPoE poprzez enkapsulację pakietów działa jako usługa uwierzytelniania, a także umożliwia szyfrowanie. Wadą PPPoE jest problem z MTU (Maximum Transmission Unit). Źle ustawiony MTU może powodować kiepską transmisję danych. Ponadto, enkapsulacja pakietów skutkuje dużym opóżnieniem, co jest widoczne podczas wysyłania pakietów ICMP.

\end{document}


