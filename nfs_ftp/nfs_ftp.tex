
\documentclass[a4paper,11pt]{article}
\usepackage{polski}
\usepackage[utf8]{inputenc}
\usepackage{amsmath}
\usepackage{amsfonts}
\usepackage{amssymb}
\usepackage{geometry}

\geometry{verbose,a4paper,tmargin=1cm,bmargin=2cm,lmargin=2cm,rmargin=2cm}

\title{\textbf{Laboratorium sieci komputerowych\\Network File System\\File Transfer Protocol}}
\author{Artur Skonecki\\Łukasz Załęski}

\begin{document}
\maketitle
\section{NFS}
\subsection{Schemat}
\begin{verbatim}
   ____                                           ____
  |    |  em3 .4                         .3 em3  |    |
  | mfs|-----------------------------------------| vf9|
  |____|             192.168.56.0/24             |____|
(serwer) (VirtualBox host-only adapter vboxnet0) (klient)

\end{verbatim}
\subsection{Konfiguracja serwera NFS}

Podczas startu systemu:

\begin{verbatim}
cat >> /etc/rc.conf << EOF
rpcbind_enable="YES"
nfs_server_enable="YES"
mountd_enable="YES"
mountd_flags="-r"
EOF
\end{verbatim}

Manualny start demonów:

\begin{verbatim}
server% rpcbind
server% nfsd -u -t -n 4
server% mountd -r
\end{verbatim}

\begin{enumerate}
\item rpcbind - pozwala klientom nfs na odkrywanie, które porty używa NFS serwer
\item mountd - demon NFS, który wykonuje polecenia nfsd
\item nfsd - demon NFS, który udostępnia usługia klientom NFS
\end{enumerate}


\begin{verbatim}
cat >> /etc/exports << EOF
/usr/share/man -ro
/home -alldirs 192.168.56.3
EOF
\end{verbatim}
Po dokonaniu modyfikacji w pliku /etc/exports należy zmusić demon mountd do ponownego wczytania konfiguracji. W tym celu należy wysłać procesowi mountd sygnał HUP:
\begin{verbatim}
/etc/rc.d/mountd reload
\end{verbatim}

Pokaż listę udostępnianych zasobów:

\begin{verbatim}
server% showmount -e
Exports list on localhost:
/home                              192.168.56.3
/usr/share/man                     Everyone
\end{verbatim}

\subsection{Konfiguracja klienta NFS}
Z lini komend:
\begin{verbatim}
mount 192.168.56.4:/usr/share/man /mnt/nfs0
mount 192.168.56.4:/home /mnt/nfs1
\end{verbatim}
Podczas startu systemu:
\begin{verbatim}
cat >> /etc/fstab << EOF
192.168.56.4:/usr/share/man	/mnt/nfs0	nfs	ro	0	0
192.168.56.4:/home	/mnt/nfs1	nfs	ro	0	0
EOF
\end{verbatim}

\section{FTP}
\subsection{Schemat}
\begin{verbatim}
   ____                                           ____
  |    |  em3 .4                         .3 em3  |    |
  | mfs|-----------------------------------------| vf9|
  |____|             192.168.56.0/24             |____|
(serwer) (VirtualBox host-only adapter vboxnet0) (klient)

\end{verbatim}
\subsection{Konfiguracja usługi FTP}

\begin{itemize}
\item /etc/ftphosts - pozwala na wyspecyfikowanie wirtualnych hostów ftp
\item /etc/ftpusers - lista użytkowników którzy nie mogą korzystać z FTP
/var/log/xferlog - domyślna lokalizacja logów ftpd w konfiguracji syslogd
\item /etc/ftpwelcome /etc/ftpmotd - drukowane po zalogowaniu użytkownika
\end{itemize}

Uruchomianie za pomocą demona inetd:

\begin{verbatim}
cat >> /etc/inetd.conf << EOF
ftp     stream  tcp     nowait  root    /usr/libexec/ftpd       ftpd -l
EOF
/etc/rc.d/inetd restart
\end{verbatim}


Podczas startu systemu. Demon ftpd bezpośrednio odpowiada na połączenia (wydajniesze niż start za pomocą inetd).

\begin{verbatim}
cat >> /etc/rc.conf << EOF
ftpd_enable="YES"
EOF
\end{verbatim}

Manualny start:

\begin{verbatim}
server% /etc/rc.d/ftpd start
\end{verbatim}

\subsection{Anonimowe logowanie}
W celu umożliwienia anonimowego logowania należy utworzyć użytkownika "ftp" bez ustawionego hasła. Wtedy możliwe będzie logowanie jako "anonymous" lub "ftp".

\subsection{Przechwycenie hasła}
W przypadku protokołu ftp przesyłane dane nie są szyfrowane. Dotyczy to także haseł.

\begin{verbatim}
client% ftp t1@192.168.56.4
Connected to 192.168.56.4.
220 test.example.com FTP server (Version 6.00LS) ready.
331 Password required for t1.
Password:

server% tcpdump -i em3 -s 100 -X port ftp
10:48:18.120259 IP 192.168.56.3.46253 > 192.168.56.4.ftp: Flags [P.], seq 10:28, ack 73, win 1040, options [nop,nop,TS val 1303188 ecr 2071881433], length 18
        0x0000:  4510 0046 0318 4000 4006 4632 c0a8 3803  E..F..@.@.F2..8.
        0x0010:  c0a8 3804 b4ad 0015 299f 3ecf 2c2c 41b8  ..8.....).>.,,A.
        0x0020:  8018 0410 48ea 0000 0101 080a 0013 e294  ....H...........
        0x0030:  7b7e 66d9 5041 5353 2074 6f6a 6573 7468  {~f.PASS.tojesth
        0x0040:  6173 6c6f 0d0a                           aslo..
\end{verbatim}

\section{Podsumowanie}

\subsection{Podsumowanie NFS}
NFS (Network File System) to sieciowy system plików, opracowany przez firmę Sun Microsystems. NFS jest najpopularniejszym rozwiązaniem na systemach UNIX-owych. NFSv3 używa TCP w warstwie transportowej. Korzyści z zastosowania NFS to obniżenie kosztów administracji, łatwienie dostępu do danych i efektywniejsze wykorzystanie zasobów.


\subsection{Podsumowanie FTP}
FTP to standardowy protokół do wymiany plików pomiędzy hostami. Standardowy protokół FTP nie jest bezpieczny - jest wrażliwy na wiele ataków. Ponadto, ruch nie jest szyfrowany. FTP pozwala na transfer plików w trybie aktywnym oraz pasywnym, który pozwala na korzystanie z usług FTP klientom za zaporą lub NAT'em.

\end{document}
